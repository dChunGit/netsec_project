For the chosen BlueKeep vulnerability, we expect to see a very wide range of vulnerable machines as the CVE is new and applicable only for older OS without a patch. At a count of almost 500 thousand Windows systems still vulnerable as reported by Shodan, a number still half of the original release, we expect to find comparable ratios of vulnerable machines in our network scans. We also expect the detected machines to be running older websites, deprecated web stacks, and similar such dated software. Additionally, these machines should be more likely to have high numbers of other possible vulnerabilities due to update neglect indicated by the lack of a BlueKeep patch. Since Windows 7 SP1 is the latest version affected by BlueKeep, we also expect to see vulnerable machines congregated in countries with poorer internet infrastructure as they will be those running a host OS released before 2012.