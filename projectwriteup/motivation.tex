At the base level, vulnerability scanners such as the one we address here exist to simply identify systems which are vulnerable to known CVEs and exploits, rather than actively identify a weakness or exploit a flaw. They are proactive in nature and give companies, individuals, and IT teams a practical tool with quantifiable outputs to better protect their own interests, data, and privacy. They pinpoint key areas of fixing, ensure successful patching, and perhaps more importantly encourage triaging of high priority threats. Additionally, many regulations and oversight committees mandate levels of security on their systems and vulnerability scanners can ensure adherence to proper standards.

Along these lines, a paper by Li, Avellino, Janies, and Collins (2016)~\cite{li_avellino_janies_collins_2016} discusses a tool designed to improve enterprise vulnerability management via detection and monitoring called the Software Asset Analyzer (SAA). This scanner addresses a few key scenarios that could lead to vulnerable systems, namely unapproved software download, failure to upgrade, lack of knowledge of installed applications, and remediation verification. To detect the potential install of vulnerable CPEs and verify it has been properly dealt with, the SAA tool takes a blacklist and checks network hosts for unauthorized CPEs by analyzing their individual configurations. By first scanning the system, the SAA tool will mark it as safe if three distinct checks are passed: all required CPEs are present, no blacklisted CPE is present, and only one version of a CPE is installed at a time. Once marked as safe, this configuration is used for future comparison analysis, thus accounting for deviations between CPE configurations on different machines. While this particular tool concentrates on categorizing the CPEs, a next step mentioned in the paper is to ``incorporate a scoring system such as CVE, NVD, and CVSS''~\cite{li_avellino_janies_collins_2016} to assess overall vulnerability of network machines. 

We will take this emphasis on analyzing for CVEs from this paper and scan for such vulnerabilities, specifically concentrating on the BlueKeep vulnerability. Developing such a network scanner enables that level of protection and oversight required to provide a secure network environment.