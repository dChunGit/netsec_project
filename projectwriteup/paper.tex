\documentclass{sig-alternate}
%\documentclass[11pt,letterpaper,singlecolumn]{article}
%%\usepackage{url}
\usepackage{epsfig}
\usepackage{subfigure}
%%\usepackage{latexsym}
%%\usepackage{times}
%%\usepackage{fancyhdr}
%%\usepackage{multirow}
%\usepackage{listing}
\usepackage{soul}
\usepackage{algorithmic}
\usepackage{algorithm}

\newcommand{\parbold}[1]{\noindent{\bf #1}}
%%%%%%%%%%%%%%%%%%%%%%%%%%
%%% Remarks
\newif\ifremark
\long\def\remark#1{
\ifremark%
        \begingroup%
        \dimen0=\columnwidth
        \advance\dimen0 by -1in%
        \setbox0=\hbox{\parbox[b]{\dimen0}{\protect\em #1}}
        \dimen1=\ht0\advance\dimen1 by 2pt%
        \dimen2=\dp0\advance\dimen2 by 2pt%
        \vskip 0.25pt%
        \hbox to \columnwidth{%
                \vrule height\dimen1 width 3pt depth\dimen2%
                \hss\copy0\hss%
                \vrule height\dimen1 width 3pt depth\dimen2%
        }%
        \endgroup%
\fi}

%%%%%%%%%%%%%%%%%%%%%%%%%%
%%% Block comments
\newcommand{\ignore}[1]{}

%%%%%%%%%%%%%%%%%%%%%%%%%%%%%%%%%%%%%
\remarktrue
%\remarkfalse
% \remark{this is a comment that shows up in text
% Switch remarkfalse on to turn comments off } 
% -- use in body, not up here
%%%%%%%%%%%%%%%%%%%%%%%%%%%%%%%%%%%%%


\begin{document}


\title{ Scanning For Vulnerable Windows RDP Hosts \\
		\large Enterprise Network Security }
\author{ Dylan Bray\\
		\texttt{ UT Austin }
		\and
		 David Chun\\
		\texttt{ UT Austin }
		\and
		}



\pagenumbering{arabic}
\date{}  % comment this out if you want the date to print

\maketitle

\begin{abstract}
Describe the overall area of contribution, the crux of the problem, 
and end with highlights of results. For the initial report, end with
the proposed experiments and what you aim to find out.
 
\end{abstract}

\section{Introduction}
\label{sec:intro}
First paragraph on the technology/society trends that lead 
to the problem at hand.

Second para: describe the key problem that if solved would make
an impact. Why the current approaches leave a gap?

Third: describe your approach. Key insight that enables your approach,
and what is novel/interesting about the insight.

Fourth, fifth: Delve deeper into the approach and experimental setup. 
In the final report, describe key findings.

End with outline or what comes next and why. 


\section{Motivation}
\label{sec:motivation}
In a paper by Li, Avellino, Janies, and Collins (2016)~\cite{li_avellino_janies_collins_2016}, they discuss a system designed to improve vulnerability management via detection and monitoring. One of the core issues covered in this paper is the case of user's failure to upgrade to a patched version of their software services, leading to increased system vulnerability. While they concentrate on categorizing the CPEs, a next step mentioned in the paper is to ``incorporate a scoring system such as CVE, NVD, and CVSS''~\cite{li_avellino_janies_collins_2016} to assess vulnerability of networks. We will take this emphasis on network vulnerability and scan for vulnerabilities, specifically concentrating on the BlueKeep vulnerability for the Windows operating system. Once we filter out and detect the vulnerable machines, we will then collect network addresses, site type, and similar holistic data to gather statistics for BlueKeep vulnerable environments in order to try to detect commonalities.

% \ignore{comment text} is better than a line comment.
\ignore{Sometimes background is merged into motivation, and is not required separately.}

\section{Our Scan}
\label{sec:arch}
Our tool will scan the network for vulnerable IPs by fingerprinting each OS, firstly by http header and secondly via methods including SYN packet fingerprinting described by Shamsi, Nadnwani, Leonard, and Loguinov (2016)~\cite{zain_ankur_derek_log_2014} as well as a TCP/IP method described by Jiao and Wu (2007)~\cite{jiao_wu_2007}. We can then proceed to filter by open ports and finally by scanning the actual host for the BlueKeep vulnerability. This scan can consist of detecting the RDP is installed and then  attempting a request/response comparison to check if the vulnerability has been patched or not. Once we have detected a BlueKeep vulnerable machine, we can then proceed to extract holistic data about our targets, including OS version, open ports, geographic location, among others. This data can then be used to aggregate common environment or network features which could give insight into the applicability, danger, and relevancy of BlueKeep.

\section{Scan Setup}
\label{sec:scan}
The actual test setup for scanning BlueKeep vulnerable machines includes 5 steps. First, we use zmap to constrain our IP search range to addresses which are responsive and have port 3389 open. We then take these discovered IPs and use scapy to send a SYN packet to a set of IP addresses, double checking for an open port 3389 to use for fingerprinting. Once we have filtered down to only a subset of the original IPs, we then use the SYN packet and the SYN/ACK reply to fingerprint the OS at the host IP address. While there are several ways of doing so, we are utilizing a passive OS fingerprinting method via scapy's p0f methods rather than an active implementation both for better accuracy as well as for simplicity. The key metrics are IP initial TTL and particularly TCP window size as they both can vary widely by OS and thus are good candidates for filtering. For example, Linux commonly has a TTL of 64 and TCP window size of 5840 while Windows XP usually has values of 128 and 65535 respectively. Windows Server/Vista/7 also vary, having TTL of 128 and TCP window size of 8192. Finally, once these candidate IPs have been identified, we will run the patch scan described above to detect whether they are truly vulnerable to the BlueKeep CVE. Those that are vulnerable proceed to our last stage where we collect data using Whois.


\section{Experimental Results}
\label{sec:results}
Our initial trial run of the fingerprinting and candidate selection step proved to match expectations, with 100\% correctly determined IP results compared to what Shodan reports as those potentially vulnerable to BlueKeep out of a sample test set of 25 IP addresses. Out of 50, we fingerprinted 93\% of the OS versions the same when compared to Shodan however, 1 of the conflicting IP addresses did not have port 3389 open at time of scanning. As previously explained, the IPs with open port 3389 and with a Windows OS Windows XP to Windows 7 are selected as our candidates for further exploration out of a larger set of IP addresses. We do this by sending IP/SYN packets via Scapy to this set of discovered IP addresses and fingerprint them with Scapy's p0f tool. The IP addresses that pass our filters can then be piped into our patch scanner to determine if they are running RDP and if so, is the machine properly patched.

However, not all potentially vulnerable machines are actually vulnerable. In our initial evaluation, the patch scanner iterated over a small list of candidate IPs and classified them into one of three categories: Vulnerable, Patched, and Offline. Vulnerable machines are shown to exhibit the properties that would allow for an attacker to successfully intrude. Patched machines are classified as all machines that respond to initial connection, but do not respond to the specially crafted packet, or force the connection closed. This behavior is likely due to firewall rules or other network-level measures intended to stop network scanning or exploitation of this specific type. Offline machines are simply those that do not respond to any packets. They may have been online during the fingerprinting phase but went offline since. About half of our candidate IPs were actually vulnerable, while the other half were patched. None were offline.

For the chosen BlueKeep vulnerability, we expect to see a more homogeneous group machines as the CVE applicable only for older OS without a patch. At a count of almost 500 thousand Windows systems still vulnerable as reported by Shodan, a number still half of the original release, we expect to find comparable ratios of vulnerable machines in our network scans, especially as BlueKeep was announced in Q1 of 2019. We also expect the detected machines to be running older websites, deprecated web stacks, and similar such dated endpoints. Additionally, these machines should be more likely to have high numbers of other possible vulnerabilities due to update neglect indicated by the lack of a BlueKeep patch. Since Windows 7 SP1 is the latest version affected by BlueKeep, we also expect to see vulnerable machines congregated in countries with poorer internet infrastructure as they will be those more likely to run a host OS released before 2012.

With such a small initial results sample, we are unable to draw conclusions about the nature of vulnerable machines. However, we are able to get relevant IP information from Whois in addition to OS information. As we gather a larger results set, we will look for patterns among the machine classifications that may prove or disprove the previously mentioned hypothesis regarding where we expect to find BlueKeep. Further, we could compare these results to existing vulnerability databases to examine what vulnerabilities are co-resident with BlueKeep.

\section{Related Work}
\label{sec:related}
Since the CVEs chosen are all related to Apache Server and more specifically versions of Apache server, it is important to determine a method of detecting which server version is actually running to correspond appropriate CVE vulnerabilities with machines on the network. This is what Shodan does to fingerprint each Apache machine on a network. However, normal methods including parsing web banners or sending invalid requests to each server may not work as banners can be forged and requests may not be handled correctly or at all. However, by crafting special requests and parsing the response, it is possible to more robustly fingerprint servers. Methods for creating these requests include detecting request methods, detecting by URL, detecting by protocol statement, and detecting by protocol version. Using a combination of these four in conjunction with the previously discussed normal methods can yield more accurate results.~\cite{jiao_wu_2007}

\section{Conclusions}
\label{sec:conclusion}
Internet scanning is incredibly important for keeping a network secure and maintained. Since BlueKeep was exposed less than a year to date, it was an ideal CVE to test not only patch adoption but also to demonstrate the value of scanners such as ours in ensuring patches were applied correctly.

Out of the initial set of 74 million IP addresses scanned, only roughly 1400 were qualified as candidate IPs, matching the correct OS and port requirements. Of this smaller subset, half of them were found to be still vulnerable to BlueKeep. Although a patch has been out for well over 6 months from time of writing, many machines either have not updated or the patch was not applied correctly.

Future work along the lines of this BlueKeep scanner would include improving OS fingerprinting to increase the scope of IPs we check while still maintaining efficiency as well as extending this tool to other potential CVEs which would rely on similar port and service parameters.


{ 
\bibliographystyle{abbrv}
\bibliography{biblio}
}

\end {document}

