The actual test setup for scanning BlueKeep vulnerable machines includes 5 steps. First, we use zmap to constrain our IP search range to addresses which are responsive and have port 3389 open. We then take these discovered IPs and use scapy to send a SYN packet to a set of IP addresses, double checking for an open port 3389 to use for fingerprinting. Once we have filtered down to only a subset of the original IPs, we then use the SYN packet and the SYN/ACK reply to fingerprint the OS at the host IP address. While there are several ways of doing so, we are utilizing a passive OS fingerprinting method via scapy's p0f methods rather than an active implementation both for better accuracy as well as for simplicity. The key metrics are IP initial TTL and particularly TCP window size as they both can vary widely by OS and thus are good candidates for filtering. For example, Linux commonly has a TTL of 64 and TCP window size of 5840 while Windows XP usually has values of 128 and 65535 respectively. Windows Server/Vista/7 also vary, having TTL of 128 and TCP window size of 8192. Finally, once these candidate IPs have been identified, we will run the patch scan described above to detect whether they are truly vulnerable to the BlueKeep CVE. Those that are vulnerable proceed to our last stage where we collect data using Whois.
