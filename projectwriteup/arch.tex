Our tool first finds open network IP addresses with port 3389 open as this is required for RDP. It then further filters these discovered IP addresses looking for vulnerable IPs by fingerprinting each OS, first by http header and second via methods including a TCP/IP/SYN method described by Han and Du (2010)~\cite{han_du_2010}. Once complete, the final step is to scan the filtered candidate IP addresses for the BlueKeep vulnerability. This scan can consist of detecting that RDP is running and then attempting a request/response comparison to check if the vulnerability has been patched or not.

The patch scan works by checking the response to a specially crafted packet, which will differ if the host is safe or vulnerable. The scanner first authenticates with the host via a handshake protocol and connects to the MS\_T120 channel. Once we send a close channel command with a specified size, there are two possible outcomes. On a vulnerable machine, the host will close the channel properly, sending a disconnect packet. However, on a patched machine, the channel will not close and no disconnect packet is sent. Thus by waiting for a couple seconds, we can determine if the host has been patched or not~\cite{dillon_2019}.

Once we detect a BlueKeep vulnerable machine, we then proceed to extract holistic data about our targets, including OS version, other open ports, geographic location, among others using Whois and other network tools. This data is then used to aggregate common environment or network features which could give insight into the applicability, danger, and relevancy of BlueKeep, including geographic location and OS version of the host machine.

The actual test setup for scanning BlueKeep vulnerable machines includes 5 steps. First, we use zmap to constrain our IP search range to addresses which are responsive and have port 3389 open. We then take these discovered IPs and use scapy to send a SYN packet to a set of IP addresses, double checking for an open port 3389 to use for fingerprinting. Once we have filtered down to only a subset of the original IPs, we then use the SYN packet and the SYN/ACK reply to fingerprint the OS at the host IP address. While there are several ways of doing so, we are utilizing a passive OS fingerprinting method via scapy's p0f methods rather than an active implementation both for better accuracy as well as for simplicity. The key metrics are IP initial TTL and particularly TCP window size as they both can vary widely by OS and thus are good candidates for filtering. For example, Linux commonly has a TTL of 64 and TCP window size of 5840 while Windows XP usually has values of 128 and 65535 respectively. Windows Server/Vista/7 also vary, having TTL of 128 and TCP window size of 8192. Finally, once these candidate IPs have been identified, we will run the patch scan described above to detect whether they are truly vulnerable to the BlueKeep CVE. Those that are vulnerable proceed to our last stage where we collect data using Whois.
