Our tool first finds open network IP addresses with port 3389 open as this is required for RDP. It then further filters these discovered IP addresses looking for vulnerable IPs by fingerprinting each OS, first by http header and second via methods including a TCP/IP/SYN method described by Han and Du (2010)~\cite{han_du_2010}. Once complete, the final step is to scan the filtered candidate IP addresses for the BlueKeep vulnerability. This scan can consist of detecting that RDP is running and then attempting a request/response comparison to check if the vulnerability has been patched or not.

The patch scan works by checking the response to a specially crafted packet, which will differ if the host is safe or vulnerable. The scanner first authenticates with the host via a handshake protocol and connects to the MS\_T120 channel. Once we send a close channel command with a specified size, there are two possible outcomes. On a vulnerable machine, the host will close the channel properly, sending a disconnect packet. However, on a patched machine, the channel will not close and no disconnect packet is sent. Thus by waiting for a couple seconds, we can determine if the host has been patched or not~\cite{dillon_2019}.

Once we detect a BlueKeep vulnerable machine, we then proceed to extract holistic data about our targets, including OS version, other open ports, geographic location, among others using Whois and other network tools. This data is then used to aggregate common environment or network features which could give insight into the applicability, danger, and relevancy of BlueKeep, including geographic location and OS version of the host machine.